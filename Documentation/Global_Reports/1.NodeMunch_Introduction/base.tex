%%%%%%%%%%%%%%%%%%%%%%%%%%
% Definition du document %
%%%%%%%%%%%%%%%%%%%%%%%%%%


\documentclass[a4paper, 12pt,table]{report}
\usepackage[utf8]{inputenc}
\usepackage[T1]{fontenc}
\usepackage[english,french]{babel}
\usepackage{packages/sleek}
\usepackage{packages/sleek-title}
\usepackage{packages/sleek-listings}
\usepackage{fancyhdr}
\usepackage{graphicx}
\usepackage{amsmath}
\usepackage{comment}
\usepackage{stackengine}
\usepackage{array}

\graphicspath{ {./images/} }

%%%%%%%%%%%%%%%%%%%%%%%%%%%%%%%%%%%%%%%%%
% Definitions des differentes commandes %
%%%%%%%%%%%%%%%%%%%%%%%%%%%%%%%%%%%%%%%%%

\pagestyle{fancy}
\renewcommand{\chaptermark}[1]{%
  \markboth{\MakeUppercase{%
  \chaptername}\ \thechapter.%
  \ #1}{}}
\renewcommand{\sectionmark}[1]{
  \markright{\thesection.\ #1}}
\fancyhf{}
\lhead[\thepage]{\leftmark}
\lfoot{\belowbaseline[0pt]{\includegraphics[scale=0.07]{iut}}}
\rfoot[\thepage]{\rightmark}
\rhead{NodeMunch}
\renewcommand{\headrulewidth}{2pt}
\renewcommand{\footrulewidth}{1pt}
\newcommand\tab[1][1cm]{\hspace*{#1}}
\usepackage{tikz}
\def\checkmark{\tikz\fill[scale=0.4](0,.35) -- (.25,0) -- (1,.7) -- (.25,.15) -- cycle;}

\tikzset{pics/.cd, check/.style={code={%
\pgfgettransformentries{\tmpxx}{\tmp}{\tmp}{\tmp}{\tmp}{\tmp}
\draw[line width=\tmpxx*1pt,draw=none,fill=blue!60] (0,.33) -- (.25,0) to 
  (0.8,.6) to (.72,.68) to (.25,.18) to (0.08,.40)-- cycle;}}}
\tikzset{pics/.cd, clipcheckmark/.style={code={% 
\pgfgettransformentries{\tmpxx}{\tmp}{\tmp}{\tmp}{\tmp}{\tmp}
\draw[line width=\tmpxx*1pt,draw=none,fill=blue!60] (0.1,.19) -- (.25,0) to 
  (0.8,.6) to (.72,.68) to (.25,.18) to (0.18,.26)-- cycle;}}}
\def\doublecheckmark{\begin{tikzpicture}[baseline={(0,0)}]
\path (0,0) pic[scale=0.4]{check} (0.16,0) pic[scale=0.4]{clipcheckmark};
\end{tikzpicture}}


%%%%%%%%%%%%%%%%%%%%%%%%%%%%%%%%%%%%%%%%%%%%%%%%%%%%%%%%%%%%%%%%%%%%%

%%%%%%%%%%%%%%%%%%%%%%
% Cahier des Charges %
%%%%%%%%%%%%%%%%%%%%%%

\begin{document}

%%%%%%%%%%%%%%%%%
% Page de Garde %
%%%%%%%%%%%%%%%%%


\logo{logo}
\institute{IUT Clermont-Ferrand}
\faculty{2ème année de BUT Informatique}

\title{NodeMunch}
\subtitle{Premier rapport}
\author{\textit{Auteurs}\\
  Baptiste \textsc{BRUNET} \tab Lola \textsc{CHALMIN}\\
  Cléo \textsc{EIRAS} \tab \tab Yann \textsc{CHAMPEAU}\\
  \tab Maxime \textsc{POINT} \tab Nicolas \textsc{BLONDEAU}
}
\date{29 Septembre 2023}

\begin{titlepage}
\maketitle
\end{titlepage}


%%%%%%%%%%%%%%%%%%%%%%
% Table des Matieres %
%%%%%%%%%%%%%%%%%%%%%%

\tableofcontents

\setlength{\parindent}{3ex}


%%%%%%%%%%%%%%%%
% Introduction %
%%%%%%%%%%%%%%%%

\chapter{Introduction}
Dans ce premier rapport de projet, nous allons procéder à une description succinte du projet NodeMunch. Nous avons sélectionné ce projet et allons, avec l'aide du Product Owner, réaliser cette application. Nous développerons ce jeu en Blazor, Kotlin, React Native, JavaScript et bien d'autres et nous nous aiderons de logiciels tels que Visual Studio, Android Studio ou encore Codefirst. L'intérêt de ce projet est d'avoir une expérience en équipe sur la conception le développement et l'intégration d'un produit multi-plateforme, mais aussi de comprendre le besoin du client et de le retranscire fidèlement. Nous travaillerons en équipe composée de six personnes et nous appliquerons la méthode SCRUM-lite afin de mener à bien ce projet ambitieux dans les meilleures conditions possibles.


%%%%%%%%%%%%%%%%%%%%%%%%%%
% Presentation du projet %
%%%%%%%%%%%%%%%%%%%%%%%%%%


\chapter{Présentation du projet}

\section{Description succinte du projet}
Notre projet a pour but de programmer une application permettant de jouer au jeu Node Kayles. Les règles du jeu sont simples :
Le plateau de jeu est un graphe. Chacun leur tour, les joueurs choisissent un sommet du graphe, et le résultat de cette action est de supprimer le sommet ainsi que tous ses voisins du graphe. Un joueur perd s'il ne peut pas jouer lors de son tour (le graphe est devenu vide).

Notre programme comportera une version mobile, une version web ainsi qu'une application de monitoring afin de traquer les performances et l'activité de notre application.

Notre application sera implémentée sur un serveur et stockera les informations dans une base de données.

Afin de permettre au joueur de jouer seul, nous développerons une intelligence artificielle capable de jouer au jeu Node Kayles.

Pour accomplir une tâche de cette ampleur dans le temps qui nous est accordé, nous agirons de façon méthodique et organisée en utilisant des outils tels que : les principes de programmation S.O.L.I.D. , les méthodes de suivi de projet SCRUM-lite, un diagramme GANTT, des KANBAN …

\section{Génèse du Projet}
Le jeu de Kayles sur les graphes, ou Node Kayles est un jeu existant indépendamment de l’application à créer pour ce projet.
Notre client, monsieur Foucaud, en a eu connaissance par différentes recherches de ses collègues dans le domaine des graphes, et sur le jeu de Kayles lui-même.
Étant adepte de jeux de réflexion stratégiques ainsi qu’intéressé par le domaine des graphes, et après n’avoir trouvé aucun jeu de Node Kayles existant, il nous a finalement sollicité dans le but de créer une application répondant à ce besoin.

\clearpage

\section{Les livrables attendus}
\begin{itemize}
	\item{Application (l'exécutable)}
	\item{MultiPlayer}
    \begin{itemize}
        \item{Application locale}
        \item{Application Distant (web)}
        \item{Application Distant Software}
    \end{itemize}
    \item{SinglePlayer}
    \begin{itemize}
        \item{Application contre une IA}
    \end{itemize}
    \item{Règles du jeu}
\end{itemize}

\section{Les contraintes}
\begin{itemize}
	\item{Contrainte de temps :}
 \begin{itemize}
     \item{La présentation générale du projet doit être livrée pour le 29 septembre 2023.}
     \item{La planification du projet devra être prête pour le 7 octobre 2023.}
     \item{Des rapports d’activités ainsi qu’une application fonctionnelle permettant de montrer notre avancée du projet au client seront à fournir pour la semaine 46, la semaine 49 et le 15 décembre 2023.}
     \item{L’application devra être livrée pour le 5 avril 2024.}
 \end{itemize}  
    \item{Contrainte de coûts :}
    \begin{itemize}
        \item{Coût potentiel du serveur pour héberger le site.}
        \item{Coûts potentiels des licences de logiciels et images utilisés.}
        \item{Coût de la rémunération des différents acteurs dans la création de l’application.}
    \end{itemize}
    \item{Contraintes matérielles :}
    \begin{itemize}
        \item{Disposition d'un poste de travail adéquat et adapté au développement}
        \item{Disposition de locaux et meubles convenables}
    \end{itemize}
\end{itemize}

\clearpage

\chapter{Conclusion}
Pour conclure, dans ce document nous avons vu en quoi consiste le projet de cette SAÉ, son objectif, les languages que nous utiliserons, les contraintes ou encore l'inspiration qui l'a motivé. \\
Le projet NodeMunch sera donc un jeu de Kayles version graphes que nous devons rendre à notre client, Mr Foucaud. Pour ceci, différentes contraintes nous serons imposées et nous devrons donc respecter la deadline finale du projet pour présenter les livrables. Nous allons tout donner et sommes très motivés à rendre la meilleure version possible de l'application qui devra coller parfaitement aux besoins transcrits par notre client.

\end{document}
